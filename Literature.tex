
\chapter{Measurments}
The LHC %and CMS -> CMS is emphasized in next sentence allow 
allows for the first detailed studies of the bottomonium family of states in ultra-relativistic heavy-ion collisions.   
Given the momentum resolution attained, and the capability of the trigger system,  CMS is well positioned to lead these studies. 
%
The measurement of bottomonium production and suppression is presented, based on the dataset collected by the CMS experiment during the 2011 PbPb collision run at  $\sqrtsnn = 2.76\TeV$. 


If a deconfined medium is formed in high-energy heavy-ion collisions, one of its most striking expected characteristics is the suppression of quarkonium states~\cite{Matsui:1986dk}. 
This takes place as the force between the constituents of the quarkonium state, a heavy quark and its antiquark, is weakened by the color screening produced by the surrounding light quarks and gluons.
%One of the most triking charateristics associated to quark gluon plasma (QGP) formation  is the suppression of quarkonium states~\cite{Matsui:1986dk}. 
%This is thought to be a direct effect of deconfinement, when the force between the constituents of a quarkonium state, a heavy quark and its antiquark, is weakened by the color screening produced by the surrounding light quarks and gluons.
 The suppression is predicted to occur above a critical temperature of the medium, and sequentially, in the
order of the \QQbar binding energy. 
Since the \PgUa is the most tightly bound state among all quarkonia, it is expected to be the one
with the highest dissociation temperature. % in the QGP.
Such a suppression pattern is expected to further depend on complications arising from additional phenomena sometimes referred to as $hot$ and $cold$ nuclear matter effects~\cite{Brambilla:2010cs,Vogt:2010aa}. 
%
The study of charmonium (\Jpsi, $\psi'$, $\chi_c$) and bottomonium ($\PgUa, \PgUb, \PgUc$, $\chi_b$) production at the unprecedented medium created at the LHC is accordingly much awaited.  
In this note, the measurements of the production and suppression of the $\PgUa$,  $\PgUb$, and  $\PgUc$ states are performed. 

%While all of charmonia as well as excited bottomonia states are expected to be suppressed in the hot and dense medium, the strongly-bound $\PgUa$ state is expected to be the last to melt-down in the QGP.
%
The production of $\PgU$(nS) states is studied by comparing their production rates in PbPb and pp collision data, taken at the same collision energy of $\sqrt{s_{NN}}=2.76\,\TeV$.
In particular, the yield of the higher-mass states is measured relative to the ground state. In this way, we explore the double ratios  --  $\PgU(2S,3S)~vs~\PgUa$ and \PbPb~$vs~\pp$ -- 
which allows a self-calibrating measurement.  
%
Several effects associated to selection, acceptance, and reconstruction mostly cancel, and only remaining factors need to be accounted for,  as corrections to the fitted ratio of raw signal yields.  


Based on the dataset collected during the first LHC PbPb run, at $\sqrtsnn = 2.76\TeV$, in 2010, and in the special $\Pp\Pp$ run at the same energy in early 2011, CMS has published first results on upsilon production and suppression in PbPb collisions. 
%
These included the first evidence for suppression of the excited $\PgU$ states relative to the ground state, at the $2.4\,\sigma$ level~\cite{prl,QM2011}. 
Suppression of the $\PgUa$ state, relative to $\Pp\Pp$ collisions at the same energy, has also been measured~\cite{QM2011,CMS_PAS_HIN-10-006}. 
These two measurements were found to be consistent with suppression of only the excited states, which result in reduced feeddown from excited to ground states. These main results may be summarized as follows:
%
\begin{linenomath}
\begin{eqnarray}
  \PgU(2S+3S)/\PgUa|_{\PbPb} & = & 0.24 _{-0.12}^{+0.13} \pm 0.02\,, \nonumber \\
  \PgU(2S+3S)/\PgUa|_{\pp} & = & 0.78 _{-0.14}^{+0.16} \pm 0.02\,,  \nonumber \\
(\chi \equiv)
  \frac{\PgU(2S+3S)/\PgUa|_{\PbPb}}{\PgU(2S+3S)/\PgUa|_{\pp}}
 & = & 0.31 _{-0.15}^{+0.19} \pm 0.03 \,,  \nonumber \\
(\raa \equiv)
 \frac{\PgUa|_{\PbPb;\, 0-20\%}}{\PgUa|_{\pp}}
& = & 0.681 \pm 0.143 \pm 0.119 \,.  \nonumber
%\label{eq:intro-rat}
\end{eqnarray}
\end{linenomath}

In the 2011 PbPb run, CMS collected a dataset approximately 20 times larger than that gathered in 2010. 
These data will be scrutinized, in order to extract further novel and precision results, during the few years ensuing datataking.
%TBD: note here current LHC PbPb run planning 
In what follows, the corresponding analysis of upsilon suppression is detailed. 
